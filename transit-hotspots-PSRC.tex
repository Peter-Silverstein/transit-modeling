% Options for packages loaded elsewhere
\PassOptionsToPackage{unicode}{hyperref}
\PassOptionsToPackage{hyphens}{url}
%
\documentclass[
]{article}
\usepackage{amsmath,amssymb}
\usepackage{iftex}
\ifPDFTeX
  \usepackage[T1]{fontenc}
  \usepackage[utf8]{inputenc}
  \usepackage{textcomp} % provide euro and other symbols
\else % if luatex or xetex
  \usepackage{unicode-math} % this also loads fontspec
  \defaultfontfeatures{Scale=MatchLowercase}
  \defaultfontfeatures[\rmfamily]{Ligatures=TeX,Scale=1}
\fi
\usepackage{lmodern}
\ifPDFTeX\else
  % xetex/luatex font selection
\fi
% Use upquote if available, for straight quotes in verbatim environments
\IfFileExists{upquote.sty}{\usepackage{upquote}}{}
\IfFileExists{microtype.sty}{% use microtype if available
  \usepackage[]{microtype}
  \UseMicrotypeSet[protrusion]{basicmath} % disable protrusion for tt fonts
}{}
\makeatletter
\@ifundefined{KOMAClassName}{% if non-KOMA class
  \IfFileExists{parskip.sty}{%
    \usepackage{parskip}
  }{% else
    \setlength{\parindent}{0pt}
    \setlength{\parskip}{6pt plus 2pt minus 1pt}}
}{% if KOMA class
  \KOMAoptions{parskip=half}}
\makeatother
\usepackage{xcolor}
\usepackage[margin=1in]{geometry}
\usepackage{graphicx}
\makeatletter
\def\maxwidth{\ifdim\Gin@nat@width>\linewidth\linewidth\else\Gin@nat@width\fi}
\def\maxheight{\ifdim\Gin@nat@height>\textheight\textheight\else\Gin@nat@height\fi}
\makeatother
% Scale images if necessary, so that they will not overflow the page
% margins by default, and it is still possible to overwrite the defaults
% using explicit options in \includegraphics[width, height, ...]{}
\setkeys{Gin}{width=\maxwidth,height=\maxheight,keepaspectratio}
% Set default figure placement to htbp
\makeatletter
\def\fps@figure{htbp}
\makeatother
\setlength{\emergencystretch}{3em} % prevent overfull lines
\providecommand{\tightlist}{%
  \setlength{\itemsep}{0pt}\setlength{\parskip}{0pt}}
\setcounter{secnumdepth}{-\maxdimen} % remove section numbering
\usepackage{titling}
\pretitle{\begin{center}\Huge\bfseries}
\posttitle{\par\end{center}}
\predate{\begin{center}\large}
\postdate{\par\end{center}}
\preauthor{\begin{center}\large}
\postauthor{\par\end{center}}
\usepackage{amsmath}
\usepackage{amsthm}
\usepackage{rotating}
\ifLuaTeX
  \usepackage{selnolig}  % disable illegal ligatures
\fi
\usepackage{bookmark}
\IfFileExists{xurl.sty}{\usepackage{xurl}}{} % add URL line breaks if available
\urlstyle{same}
\hypersetup{
  pdftitle={Transit Usage in Seattle: A Spatial Investigation},
  pdfauthor={Peter Silverstein},
  hidelinks,
  pdfcreator={LaTeX via pandoc}}

\title{Transit Usage in Seattle: A Spatial Investigation}
\author{Peter Silverstein}
\date{2024-12-09}

\begin{document}
\maketitle

\begin{center}
    {\large Final Project}\\[0.5cm]
    {\large GIS and Spatial Analysis}\\[0.5cm]
    {\large QMSS5070}\\[0.5cm]
\end{center}

\newpage

\section{Introduction}\label{introduction}

\subsection{Research Question:}\label{research-question}

\begin{enumerate}
\def\labelenumi{\arabic{enumi}.}
\tightlist
\item
  How does transit usage percentage (percent of trips using mass transit
  / total trips per census tract) vary spatially in and around Seattle
  and Tacoma, Washington?
\item
  How does this variation relate to population density, median income,
  and median age at the census tract level?
\end{enumerate}

\subsection{Purpose of Study:}\label{purpose-of-study}

There are essentially two purposes to this study. The first is to better
understand where there are concentrations of high and low transit usage
around the region. If there is clustering and we see hotspots and
coldspots, further policy-focused questions can be asked. For example:
given clustering, what characteristics of a census tract makes in more
or less likely to be in one of these hot or cold zones? How might we
allocate resources across hot zones, cold zones, and those in-between to
increase the adoption of transit by commuters? Is the dispersion of
transit availability closely related to the demand and does the
dispersion favor certain demographic groups over others?

The second research question is a very basic attempt at answering one of
these follow-up questions. By understanding how the three variables
(population density, median income, and median age) are related to the
outcome of interest (percentage of commuter trips taken using public
transit), we can begin to fill in the knowledge gaps demonstrated by the
questions above.

\subsection{Hypotheses}\label{hypotheses}

\begin{enumerate}
\def\labelenumi{\arabic{enumi}.}
\tightlist
\item
  I believe we will see transit hotspots close to urban centers (e.g.,
  Seattle and Tacoma, the two biggest cities in the region of interest).
  Further, I believe the opposite will be true for coldspots--they
  should exist further outside urban centers. These ideas are based on
  the fact that transit lines themselves tend to be clustered in
  high-density, urban areas, meaning opportunities for mass transit
  travel are more convenient and plentiful in more central urban areas.
\item
  I expect that transit use percentage is positively associated with
  population density and median income and negatively associated with
  age. I make this hypothesis about population density based on the
  reasoning above. I expect younger people to (a) be more likely to live
  in highly urban areas and (b) be less likely to own a personal vehicle
  (such as a car). Of the three variables, I am the least confident
  about median income, because I think the relationship could be pulled
  in both positive and negative directions. On one hand, urban areas
  tend to be more expensive and thus have a higher requirement for
  income to live there. On the other hand, lower income should be
  associated with lower rates of car ownership and thus lower income
  would be associated with higher transit ridership.
\end{enumerate}

\section{Data and Methodology:}\label{data-and-methodology}

\subsection{Data Sources:}\label{data-sources}

\begin{enumerate}
\def\labelenumi{\arabic{enumi}.}
\tightlist
\item
  The \textbf{Puget Sound Regional Association (PSRC) Household Travel
  Survey (HTS) 2017-23} is a biennial survey of commuters done in the
  King, Kitsap, Pierce, and Snohomish counties of Washington state (the
  counties surrounding Seattle and Tacoma). The present analysis uses
  the Trips dataset from the HTS. Each observation in the dataset
  represents a single trip taken by a respondent and includes a variety
  of variables. Most important for my analysis are origin/destination
  tract,
\end{enumerate}

\end{document}
